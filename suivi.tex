\documentclass[a4paper, 12pt, french]{article}

\usepackage[utf8]{inputenc} %package pour le français sous ubuntu : à vous d'adapter
\usepackage[french]{babel}  %pour le français
\usepackage[T1]{fontenc}    %pour les polices
\usepackage{fullpage}

\title{Cours de statistiques\\Principe de la validation croisé\\Cahier de suivi}
\author{Etienne \bsc{Batise}, Thibaud \bsc{Dauce}}

\begin{document}
   \maketitle 
   %% Résumé du sujet, pourquoi on l'a choisi
   \section{Résumé} 
   Le principe de la validation croisée est une importante méthode statistiques permettant d'estimer la fiabilité d'un échantillon par rapport à un modèle donné. Il s'agit d'une méthode puissante dont la mise en oeuvre dans le domaine informatique implique généralement du machine-learning. 

   Nous avons choisi ce sujet car dans différents projets nous devons évaluer la qualité d'une donnée avant de l'enregistrer et il est compliqué de mettre en place une évaluation dynamique. Après quelques première recherche sur le sujet, nous nous sommes aperçus que le principe de validation croisé correspond en partie à nos attentes.

   \section{Plan} 
   Ce document a pour but de présenter notre organisation et notre méthodologie de travail au cours du projet. 

   Voici le plan de que nous allons suivre durant la présentation:
   \begin{enumerate}
       \item Introducion \textit{- Mise en évidence du problème statistique}
       \item Théorie \textit{- Définitions, lois, théorèmes}
       \item Pratique \textit{- Démonstration d'un code sur l'outil MatLab}
       \item Plus loin \textit{- Explication de l'utilité pratique}
   \end{enumerate}
   
   \section{Répartition des tâches}

   \begin{description}
       \item[Etienne \bsc{Batise}: ]Théorie - Démo
       \item[Thibaud \bsc{Dauce}: ]Pratique - Démo
   \end{description}
\end{document}